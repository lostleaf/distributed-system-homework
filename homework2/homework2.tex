%%%%%%%%%%%%%%%%%%%%%%%%%%%%%%%%%%%%%%%%%%%%%%%%%%%%%%%%%%%%%%%%%%%%%%%%
\documentclass[12pt]{article}
\usepackage{amssymb,amsthm}
\usepackage{amsmath,amssymb,CJK}
\usepackage{graphicx}
\usepackage{subfigure}
\usepackage{listings}
\usepackage{enumerate}
 
\openup 7pt\pagestyle{plain} \topmargin -40pt \textwidth
14.5cm\textheight 21.5cm
\parskip .09 truein
\baselineskip 4pt\lineskip 4pt \setcounter{page}{1}
\def\a{\alpha}\def\b{\beta}\def\d{\delta}\def\D{\Delta}\def\fs{\footnotesize}
\def\g{\gamma}
\def\G{\Gamma}\def\l{\lambda}\def\L{\Lambda}\def\o{\omiga}\def\p{\psi}
\def\se{\subseteq}\def\seq{\subseteq}\def\Si{\Sigma}\def\si{\sigma}\def\vp{\varphi}\def\es{\varepsilon}
\def\sc{\scriptstyle}\def\ssc{\scriptscriptstyle}\def\dis{\displaystyle}
\def\cl{\centerline}\def\ll{\leftline}\def\rl{\rightline}\def\nl{\newline}
\def\ol{\overline}\def\ul{\underline}\def\wt{\widetilde}\def\wh{\widehat}
\def\rar{\rightarrow}\def\Rar{\Rightarrow}\def\lar{\leftarrow}
\def\Lar{\Leftarrow}\def\Rla{\rightleftarrow}\def\bs{\backslash}
\def\ra{\rangle}\def\la{\langle}\def\hs{\hspace*}\def\rb{\raisebox}
\def\ni{\noindent}\def\hi{\hangindent}\def\ha{\hangafter}
\def\vs{\vspace*}
\def\hom#1{\mbox{\rm Hom($#1,#1$)}}
\def\thebeg{\vskip8pt\par\ni}
\def\theend{\vskip5pt\par}
\def\chabeg{\pagebreak\par}
\def\chaend{\vskip20pt\par}
\def\secbeg{\vskip15pt\par}
\def\secend{\vskip10pt\par}
\def\exebeg{\vskip10pt}
\def\exeend{\vskip6pt}
\def\undot#1{\mbox{$\mbox{#1}\hs{-1.5ex}_{_{\dis\centerdot}}\,\,$}}
\def\qed{\hfill\mbox{$\Box$}}
\def\C{\mathbb{C}}
\def\Q{\mathbb{Q}}
\def\ii{\mbox{\,{\bf i}\,}}
\def\jj{\mbox{\,{\bf j}\,}}
\def\AA{{\cal A}}
\def\BB{{\cal B}}
\def\DD{{\cal D}}
\def\EE{{\mbox{\bf 1}}}
\def\OO{{\mbox{\bf 0}}}
\def\kk{{\mbox{\bf k}}}
\def\R{\mathbb{R}}
\def\F{\mathbb{F}{\ssc\,}}
%\def\similar{\rb{-4pt}{\mbox{\,\~\,}}}
\def\similar{\backsim}
\def\Llra{\Longleftrightarrow}
\def\Lra{\Longrightarrow}
\def\Lla{\Longleftarrow}
\def\mat#1#2{\mbox{$\left(\begin{array}{#1}#2\end{array}\right)$}}
\def\det#1#2{\mbox{$\left|\begin{array}{#1}#2\end{array}\right|$}}
\def\eset{\emptyset}
\parindent=5ex
\setlength{\parindent}{0pt}
\setlength{\parskip}{1ex plus 0.5ex minus 0.2ex}
\newtheorem{Example}{\text{例}}

%\begin{CJK*}{UTF8}{gbsn}
 
\date{}
\title{Homework2}
\author{Qinglin Li, 5110309074}
\begin{document}
\maketitle

\section*{Problem 1}
\subsection*{Protocol:}
\subsubsection*{Round $1$}
The sender sends message $m$ to everybody
\subsubsection*{Round $2$---Round $f+2$}
The sender and all processes that have received the message $m$ propose for $m$, others propose for \texttt{SF}\\
Run the consensus protocol with the decision function
$
	f(V)=\left\{
	\begin{aligned}
		&m  &(m\in V)\\
		&SF &(m\ V)
	\end{aligned}
	\right.
$
\subsection*{Proof:}
\subsubsection*{Termination}
Since the consensus protocol terminates, this protocol also terminates.
\subsubsection*{Agreement}
If a correct process have delivered $m$, it must have decided $m$ in the consensus protocol.\\
According to the agreement of the consensus protocol, all correct processes must have decided $m$. So all correct processes have delivered $m$.
\subsubsection*{Validity}
According to the validity of the consensus protocol, if the sender is correct, all correct processes should eventually have decided $m$ and delivered $m$;
\subsubsection*{Integrity}
According to the consensus protocol, every correct process delivers at most one message.\\
If a correct process have delivered $m$, it must have decided $m$ in the consensus protocol. According to the integrity of the consensus protocol, there must be some process have proposed $m$ which is sent form the sender.

\section*{Problem 2}
\subsection*{Protocol:}
\subsubsection*{Round $1$}
The sender sends message $m$ to everybody
\subsubsection*{Round $2$---Round $f+1$}
The sender sends message $m$ to everybody.\\
The sender and all processes that have received the message $m$ propose for $m$, others propose for \texttt{SF}\\
Run the consensus protocol for $f$ round as if there are at most $f-1$ processes crash with the decision function
$
	f(V)=\left\{
	\begin{aligned}
		&m  &(m\in V)\\
		&SF &(m\ V)
	\end{aligned}
	\right.
$
\subsection*{Proof:}
\subsubsection*{Termination}
According to the termination of the consensus protocol, this protocol must terminate.
\subsubsection*{Agreement}
If the sender hadn't crashed until round $2$, every correct process should have received $m$. So every correct process should decide $m$ and then deliver $m$.\\
If the sender crashed before the second round, there are at most $f-1$ faults remaining. So the sonsensus protocol can ensure agreement.
\subsubsection*{Validity}
If the sender is correct, all correct processes must have received the message $m$ from the sender in the second round. So all of them deliver $m$.
\subsubsection*{Integrity}
The decision function $f$ ensures every correct process delivers at most one $1$ message. If it delivers $m\neq SF$, then sender must have broadcast $m$

\section*{Problem 3}
\begin{enumerate}
\item
\textbf{\large Protocol:}\\

Suppose sender sends the message $m$ and $p$ is an arbitrary process.\\
In round $k(1\leq k\leq t+1)$:
\begin{enumerate}
\item If $p$ is the sender or $p$ has received $m$ in the previous rounds, $p$ sends $m$ to all.
\item $p$ receives messages sent in round $k$.\\
\end{enumerate}
In round $t + 2$:
\begin{enumerate}
\item If $p$ is sender or $p$ has received $m$ in the previous rounds,$p$ sends $m$ to all. Otherwise, $p$ sends $SF$ to all.
\item $p$ receives messages sent in round $t+2$.
\item If $p$ have received some message $m$ for more than $\frac{n}{2}$ times, $p$ delivers $m$.
\end{enumerate}
\newpage
\textbf{\large Proof:}\\

\textbf{Termination}\\

Since the first $t+1$ rounds are same as the TRB protocol except for message dilivering, all correct processes should send the same message in round $t+2$ and then they eventually deliver the message.\\

\textbf{Uniform Agreement}\\

If a process $p$ delivered $m\neq SF$ , it must have received $m$ for more than $\frac{n}{2}$ times in round $t + 2$. So more than $\frac{n}{2}$ processes sent $m$ in round $t+2$ and they must have received $m$ before round $t+2$.\\
Since the first $t+1$ rounds are same as the TRB protocol except for message dilivering, by the end of round t + 1, all correct processes should have received $m$.\\
So all correct processes should deliver $m$ in round $t + 2$.\\

If a process $p$ delivered $m=SF$ , it must have received $SF$ for more than $\frac{n}{2}$ times in round $t + 2$. So more than $\frac{n}{2}$ processes sent $SF$ in round $t+2$ and they must haven't received $m$ before round $t+2$.\\
Since the first $t+1$ rounds are same as the TRB protocol except for message dilivering, by the end of round t + 1, all correct processes should have received $m$ if any correct process have reveived $m$.\\
So none of the correct process have ever received $m$ and all correct processes should deliver $SF$ in round $t + 2$.\\

\textbf{Validity}\\

If the sender is correct, all correct processes should have received $m$.\\
So in round $t + 2$ every correct process should have received $m$ for more than $\frac{n}{2}$ times because $n>2t$.\\
All correct processes delivered $m$\\

\textbf{Integrity}\\

In round t + 2, every correct process should deliver at most one message.\\
If a correct process delivered m, then the sender must have sent m.\\

\item
\textbf{\large Proof:}\\

(By contradiction)\\
Suppose there is an protocol with $n\leq 2t$\\
The processes can be divided to two groups $X$ and $Y$ whose size are less than or equal to $t$. WLOG, let the sender in $X$.\\
If all processes in $X$ crashed at the beginning, all the processes in $Y$ should deliver $SF$.\\
But if all processes in $X$ are correct and all process in $Y$ cannot receive messages. Because these two situations are similar for every process in $Y$, they should deliver $SF$. However, every process in $X$ should deliver $m$.\\
Contradiction to \textbf{uniform agreement}.\\
\end{enumerate}

\section*{Problem 4}

\subsection*{Termination}
Since the vector $V_p$ have at least one element that is not $\bot$, every process should eventually decide.\\

\subsection*{Agreement}
According to Lemma $2$, for every correct process $p$, $V_p$ are the same in the end. So the decisions
must be the same.

\subsection*{Integrity}
At the end of the protocol, every correct process should decide only one value.\\
If the value $v$ is decided, there must be some process $p$ having $v$ as element of $V_p$ by definition of the operations.\\

\subsection*{Validity}
If all process propose for $v$, the only possible value except for $\bot$ in every $V_p$ at the end is $v$.\\
So every correct process should eventually decide $v$.

\section*{Problem 5}
Suppose $n=3$ and $f=1$\\
\begin{tabular}{|c|c|c|c|}
\hline 
• & $p_1$ & $p_2$ & $p_3$ \\ 
\hline 
input & $0$ & $1$ & $1$ \\ 
\hline 
sent a-value & $0$ & $1$ & $1$ \\ 
\hline 
received a-values & $(0,1)$ & $(1,1)$ & $(1,1)$ \\ 
\hline 
sent b-value & $\bot$ & $1$ & $1$ \\ 
\hline 
received b-value & $(1,\bot)$ & $(1,\bot)$ & $(1,1)$ \\ 
\hline 
decision & none & none & 1 \\ 
\hline 
\end{tabular} 

Suppose then $p_1$ and $p_2$ have $0$ as their a-value and $p_3$ have a really large latency so that $p_1$ and $p_2$ always ignore its messages.\\
So $p_1$ and $p_2$ would decide $0$\\
Agreement is broken.\\
\end{document}
